%
% sample root file for your "contribution" to a contributed volume
%
% Use this file as a template for your own input.
%
%%%%%%%%%%%%%%%% Springer %%%%%%%%%%%%%%%%%%%%%%%%%%%%%%%%%%
\documentclass[graybox,footinfo]{svmult}

\smartqed
\usepackage{mathptmx}       % selects Times Roman as basic font
\usepackage{helvet}         % selects Helvetica as sans-serif font
\usepackage{courier}        % selects Courier as typewriter font
\usepackage{type1cm}        % activate if the above 3 fonts are
% not available on your system
\usepackage{graphicx}       % standard LaTeX graphics tool
% when including figure files

\usepackage{array,colortbl}
\usepackage{amsmath,amsfonts,amssymb,bm} % no amsthm, Springer defines Theorem, Lemma, etc themselves
%\usepackage[mathx]{mathabx}
\DeclareFontFamily{U}{mathx}{\hyphenchar\font45}
\DeclareFontShape{U}{mathx}{m}{n}{
	<5> <6> <7> <8> <9> <10>
	<10.95> <12> <14.4> <17.28> <20.74> <24.88>
	mathx10
}{}
\DeclareSymbolFont{mathx}{U}{mathx}{m}{n}
\DeclareFontSubstitution{U}{mathx}{m}{n}
\DeclareMathAccent{\widecheck}{0}{mathx}{"71}



% Note that Springer defines the following already:
%
% \D upright d for differential d
% \I upright i for imaginary unit
% \E upright e for exponential function
% \tens depicts tensors as sans serif upright
% \vec depicts vectors as boldface characters instead of the arrow accent
%
% Additionally we throw in the following common used macro's:
\input{macros}

% Macros below are now included in macros.tex from MCQMC 2016 web site
% This spot formerly included macros that are now in macros.tex

% indicator boldface 1:
\DeclareSymbolFont{bbold}{U}{bbold}{m}{n}
\DeclareSymbolFontAlphabet{\mathbbold}{bbold}
%\newcommand{\ind}{\mathbbold{1}}


\usepackage{microtype} % good font tricks

\usepackage[colorlinks=true,linkcolor=black,citecolor=black,urlcolor=black]{hyperref}
\urlstyle{same}
\usepackage{bookmark}
\pdfstringdefDisableCommands{\def\and{, }}
\makeatletter % to avoid hyperref warnings:
\providecommand*{\toclevel@author}{999}
\providecommand*{\toclevel@title}{0}
\makeatother



\usepackage{bbm,mathtools,array,longtable,booktabs,graphicx,color,enumitem,xspace}
%\input FJHDef.tex


\newcommand{\QMCPYnorm}[2][{}]{\ensuremath{\left \lVert #2 \right \rVert}_{#1}}
\newcommand{\QMCPYnormnorm}[2][{}]{\ensuremath{\lVert #2 \rVert}_{#1}}
\newcommand{\QMCPYbignorm}[2][{}]{\ensuremath{\bigl \lVert #2 \bigr \rVert}_{#1}}
\newcommand{\QMCPYBignorm}[2][{}]{\ensuremath{\Bigl \lVert #2 \Bigr \rVert}_{#1}}
\newcommand{\QMCPYabs}[1]{\ensuremath{{\left \lvert #1 \right \rvert}}}
\newcommand{\QMCPYbigabs}[1]{\ensuremath{{\bigl \lvert #1 \bigr \rvert}}}
\newcommand{\QMCPYBigabs}[1]{\ensuremath{{\Bigl \lvert #1 \Bigr \rvert}}}
\newcommand{\QMCPYbiggabs}[1]{\ensuremath{{\biggl \lvert #1 \biggr \rvert}}}


\providecommand{\HickernellFJ}{Hickernell}

\definecolor{orange}{rgb}{1.0,0.3,0.0}
\definecolor{violet}{rgb}{0.75,0,1}
\newcommand{\frednote}[1]{  {\textcolor{red}  {\mbox{**Fred:} #1}}}
\newcommand{\scnote}[1]{ {\textcolor{blue}  {\mbox{**SC:} #1}}}
\newcommand{\tonynote}[1]{ {\textcolor{orange}  {\mbox{**Tony:} #1}}}

%\journal{Journal of Complexity}

\allowdisplaybreaks[4]

%   python styling
\usepackage[T1]{fontenc}
\usepackage{lmodern}
\usepackage{listings}
\usepackage{xcolor}
\definecolor{darkgreen}{rgb}{0,0.6,0}
\lstdefinestyle{Python}{
    showstringspaces=false,
    language        = Python,
    basicstyle      = \small\ttfamily,
    morekeywords = {as},
    keywordstyle    = \color{blue},
    stringstyle     = \color{darkgreen},
    commentstyle    = \color{darkgreen}\ttfamily,
	breaklines = true,
	postbreak=\text{$\hookrightarrow$\space},
	% style >>> and ... 
	%   see: https://tex.stackexchange.com/questions/326655/make-a-keyword-in-listings-enviorment
	alsoletter = {>,.} ,
    morekeywords = [2]{>>>,...},
    keywordstyle = [2]\color{cyan}\bfseries}

\newcommand{\AGSComment}[1]{{\color{cyan} Aleksei: #1}}
\newcommand{\FJHComment}[1]{{\color{magenta} Fred: #1}}
\newcommand{\JRComment}[1]{{\color{violet} Jag: #1}}

\newcommand{\hmu}{\widehat{\mu}}
\newcommand{\IID}{\textup{IID}}
\newcommand{\LD}{\textup{LD}}
\newcommand{\IMP}{\textup{IS}}
\newcommand{\unif}{\textup{unif}}
\newcommand{\IIDsim}{\overset{\IID}{\sim}}
\newcommand{\LDsim}{\overset{\LD}{\sim}}
\newcommand{\Msim}{\overset{\textup{M}}{\sim}}
\newcommand{\dig}{\textup{dig}}
\newcommand{\mA}{\mathsf{A}}
\newcommand{\mSigma}{\mathsf{\Sigma}}
\newcommand{\mLambda}{\mathsf{\Lambda}}
\newcommand{\mI}{\mathsf{I}}
\newcommand{\cube}{[0,1]^d}
\newcommand{\Kum}{\textup{Kum}}

\begin{document}

\title*{Quasi-Monte Carlo Software}
\authorrunning{S.-C.\ T.\ Choi et al}
\author{Sou-Cheng T. Choi \and Fred J. Hickernell \and R. Jagadeeswaran \and Michael J. McCourt \and Aleksei G. Sorokin}
\institute{Sou-Cheng T. Choi \at Department of Applied Mathematics, Illinois Institute of Technology,\\ RE 220, 10 W.\ 32$^{\text{nd}}$ St., Chicago, IL 60616; and Kamakura Corporation, 2222 Kalakaua Ave, Suite 1400, Honolulu, HI 96815 \email{schoi32@iit.edu}
\and
Fred J. Hickernell \at Center for Interdisciplinary Scientific Computation and \\
Department of Applied Mathematics, Illinois Institute of Technology \\ RE 220, 10 W.\ 32$^{\text{nd}}$ St., Chicago, IL 60616 \email{hickernell@iit.edu}
\and
R. Jagadeeswaran \at
Department of Applied Mathematics, Illinois Institute of Technology,\\ RE 220, 10 W.\ 32$^{\text{nd}}$ St., Chicago, IL 60616 \email{jrathin1@iit.edu}; and \\Wi-Tronix LLC, 631 E Boughton Rd, Suite 240, Bolingbrook, IL 60440
\and
Michael J. McCourt \at SigOpt, an Intel company, \\
100 Bush St., Suite 1100, San Francisco, CA 94104
\email{mccourt@sigopt.com}
\and 
Aleksei G. Sorokin \at
Department of Applied Mathematics, Illinois Institute of Technology,\\ RE 220, 10 W.\ 32$^{\text{nd}}$ St., Chicago, IL 60616 \email{asorokin@hawk.iit.edu}}

\maketitle

\abstract{Practitioners wishing to experience the efficiency gains from using low discrepancy sequences need correct, well-written software.  This article, based on our MCQMC 2020 tutorial, describes some of the better quasi-Monte Carlo (QMC) software available.  We highlight the key software components required to approximate multivariate integrals or expectations of functions of vector random variables by QMC.  We have combined these components in QMCPy, a Python open source library, which we hope will draw the support of the QMC community.  Here we introduce QMCPy.}


\section{Introduction} \label{sec:intro}
Quasi-Monte Carlo (QMC) methods promise great efficiency gains over independent and identically distributed (IID) Monte Carlo (MC) methods.  In some cases, QMC  achieves one hundredth of the error of IID MC in the same amount of time (see Figure \ref{fig:sc_comp}). Often, these efficiency gains are obtained simply by replacing IID sampling with low discrepancy (LD) sampling, which is the heart of QMC. 

As a practitioner, you might wish to test whether QMC would speed up your computation.  You would like to easily access the best QMC algorithms available.  As a theoretician or an algorithm developer, you might want to demonstrate your best ideas on  various  use cases to show their practical value.  

This tutorial points to some of the best QMC software available.  Then we describe  QMCPy \cite{QMCPy2020a}, which is crafted to be a community owned Python library that combines the best QMC algorithms and interesting use cases from various authors under a common user interface.

The model problem for QMC is approximating a multivariate integral,
\begin{equation} \label{eq:integral}
	\mu := \int_\calT g(\bst) \, \lambda(\bst) \, \D \bst,
\end{equation}
where $g$ is the integrand, and $\lambda$ is a non-negative weight.  If $\lambda$ is a probability distribution (PDF) for the random variable $\bsT$, then $\mu$ is the mean of $g(\bsT)$.  Regardless, we perform a suitable variable transformation to interpret this integral as the  mean of a function of a multivariate, standard uniform random variable:
\begin{equation} \label{eq:fintegral}
	\mu = \bbE[f(\bsX)] =  \int_{\cube}  f(\bsx) \,  \D \bsx , \qquad \bsX \sim \calU\cube.
\end{equation}

QMC approximates the population mean, $\mu$,  by a sample mean,
\begin{equation} \label{eq:samplemean}
	\hmu := \frac 1n \sum_{i=0}^{n-1} f(\bsX_i), \qquad \bsX_0, \bsX_1, \ldots \Msim \calU\cube.
\end{equation}
The choice of this sequence and the choice of $n$ to satisfy  the prescribed error requirement,
\begin{equation} \label{eq:err_req}
	\QMCPYabs{\mu - \hmu} \le \varepsilon \qquad \text{absolutely or with high probability},
\end{equation} 
are important decisions, which  QMC software helps the user make.

Here, the notation $\Msim$ means that the sequence mimics the specified, target distribution, but not necessarily in a probabilistic way.  We  use this notation in two forms:  $\IIDsim$ and $\LDsim$.

IID sequences must be random. The position of any point is not influenced by any other, so clusters and gaps occur.  A randomly chosen subsequence of an IID sequence is also IID.  When we say that $\bsX_0, \bsX_1, \ldots \IIDsim F$ for some distribution $F$, we mean that for any positive integer $n$, the  multivariate probability distribution of $\bsX_0, \ldots, \bsX_{n-1}$ is the product of the marginals, specifically,
\begin{equation*}
	F_{n}(\bsx_0, \ldots, \bsx_{n-1}) = F(\bsx_0) \cdots  F(\bsx_{n-1}).
\end{equation*}
When IID points are used to approximate $\mu$ by the sample mean, the root mean squared error is $\calO(n^{-1/2})$.  Figure \ref{fig:comparePts} displays IID uniform points, $\bsX^{\IID}_0, \bsX^{\IID}_1, \ldots \IIDsim \calU[0,1]^2$, i.e.,  the target distribution is $F_\unif(\bsx) = x_1 x_2$.


\begin{figure}[t]
	\includegraphics[height=5.8cm]{QMCSoftwareArticle/figs/dd_iid_uniform_pts.png}
	\quad
	\includegraphics[height=5.8cm]{QMCSoftwareArticle/figs/dd_sobol_pts.png}
	\caption{IID points (left) contrasted with LD points (right).  The LD points cover the square more evenly.}
	\label{fig:comparePts}
\end{figure}

LD sequences may be deterministic or random, but each point is carefully coordinated with the others so that they fill the domain well.  Subsequences of LD sequences are generally not LD.  When we say that $\bsX_0, \bsX_1, \ldots \LDsim \calU\cube$, we mean that for any positive integer $n$,  the \emph{empirical distribution} of $\bsX_0, \ldots, \bsX_{n-1}$, denoted $F_{\{\bsX_i\}_{i=0}^{n-1}}$,  approximates the uniform distribution, $F_\unif$, well (relative to $n$).  (The empirical distribution of a set assigns equal probability to each point.)  The measure of the difference between the empirical distribution of a set of points and the uniform distribution is called a \emph{discrepancy} and is denoted $D(\{\bsX_i\}_{i=0}^{n-1})$ \cite{DicEtal14a,Hic97a,Hic99a,Nie92}.  This is the origin of the term ``low discrepancy'' points or sequences.  LD points by definition have a smaller discrepancy than IID points.  Figure \ref{fig:comparePts} contrasts IID uniform points with LD points, $\bsX^{\LD}_0, \bsX^{\LD}_1 \ldots \LDsim \calU[0,1]^2$, in this case linearly scrambled Sobol' points.

The error in using the sample mean to approximate the integral can be bounded according to the Koksma-Hlawka inequality and its extensions \cite{DicEtal14a,Hic97a,Hic99a,Nie92} as the product of the discrepancy of the sampling sequence and the variation of the integrand, denoted $V(\cdot)$:
\begin{equation}
	\QMCPYabs{\mu - \hmu} = \QMCPYbiggabs{\int_{\cube} f(\bsx) \, \D (F_\unif - F_{\{\bsX_i\}_{i=0}^{n-1}}) (\bsx)} \le D\bigl(\{\bsX_i\}_{i=0}^{n-1}\bigr) V(f),
\end{equation} 
The variation is a (semi-) norm of the integrand in a suitable Banach space.  The discrepancy corresponds to the norm of the error functional for that Banach space.  For typical Banach spaces, the discrepancy of LD points is $\calO(n^{-1+\epsilon})$, which is a higher convergence order than for IID points.  For details, the reader is referred to the references.  Here, we expect the reader to see in Figure \ref{fig:comparePts} that the LD points cover the integration domain more evenly than IID points.  LD sampling can be thought of as highly stratified sampling.  In the examples below the reader will see the demonstrably smaller cubature errors arising from using LD points.

In the sections that follow we first overview available QMC software.  We next describe an architecture for good QMC software, i.e., what are the key components and how should they interact.  We then describe how we have implemented this architecture in QMCPy.  Finally, we summarize further directions that we hope QMCPy and other QMC software projects will take.  Those interested in following or contributing to the development of QMCPy are urged to visit the GitHub repository at \href{https://github.com/QMCSoftware/QMCSoftware}{\nolinkurl{https://github.com/QMCSoftware/QMCSoftware}}.

We have endeavored to be as accurate as possible at the time of writing this article.  We hope that progress in QMC software development will make this article somewhat obsolete in a good way in the next several years.

\section{Available Software for QMC} \label{sec:available} 
QMC software spans  LD sequence generators, cubatures, and applications.  We review the better known software collections, recognizing that some software overlaps multiple categories.
Software focusing on generating high quality LD sequences  and their generators includes
\begin{description}[format=\textup,format=\textbf]
% https://tex.stackexchange.com/questions/74279/how-to-add-bullets-to-description-lists
	\item[BRODA] Sobol' sequences in C, MATLAB, and Excel \cite{BRODA20a},
	\item[Burkhardt] Various QMC software in C++, Fortran, MATLAB, and Python~\cite{Bur20a},
	\item[LatNet Builder] Generating vectors/matrices for lattices and digital nets~\cite{LEcEtal22a,LatNet},
	\item[MATLAB] Sobol' and Halton sequences, commercial \cite{MAT9.9},
	\item[MPS] Magic Point Shop, lattices and Sobol' sequences \cite{Nuy17a},
	\item[Owen] Randomized Halton sequences in R \cite{Owe20a},
	\item[PyTorch] Scrambled Sobol' sequences \cite{PyTorch},
	\item[QMC.jl] LD Sequences in Julia \cite{Rob20a}, and
	\item [qrng]  Sobol', Halton, and Korobov sequences in R \cite{QRNG2020}.
\end{description}
Software focusing on QMC cubatures and applications includes
\begin{description}[format=\textup,format=\textbf]
	\item[GAIL] Automatic (Q)MC stopping criteria in MATLAB \cite{ChoEtal20a},
	\item[ML(Q)MC] Multi-Level (Q)MC routines in C++, MATLAB, Python, and R~\cite{GilesSoft},
	\item[OpenTURNS] Open source initiative for the Treatment of Uncertainties, Risks 'N Statistics in Python \cite{OpenTURNS},
	\item[QMC4PDE] QMC for elliptic PDEs with random diffusion coefficients \cite{KuoNuy16a},
	\item[SSJ] Stochastic Simulation in Java \cite{SSJ}, and
	\item[UQLab] Framework for Uncertainty Quantification in MATLAB \cite{UQLab2014}.
\end{description}

The sections that follow describe QMCPy \cite{QMCPy2020a}, which is our attempt to combine some of the best of the above software under a common user interface written in Python 3.  The choice of language was determined by the desire to make QMC software accessible to a broad audience, especially the tech industry.

\section{Components of QMC Software}
QMC cubature can be summarized as follows.  We want to approximate the expectation, $\mu$, well by the sample mean, $\hmu$, where \eqref{eq:integral}, \eqref{eq:fintegral}, and \eqref{eq:samplemean} combine to give
\begin{multline} \label{eq:cubSummary}
	\mu : = \int_\calT g(\bst) \, \lambda(\bst) \, \D \bst  = \bbE[f(\bsX)] = \int_{\cube} f(\bsx) \, \D \bsx \approx \frac 1n \sum_{i=0}^{n-1} f(\bsX_i) =: \hmu, \\
	 \bsX \sim \calU\cube, \ \bsX_0, \bsX_1, \ldots \Msim \calU\cube.
\end{multline}
Moreover, we want to satisfy the error requirement in \eqref{eq:err_req}.
This requires four components, which we implement as QMCPy classes.

\begin{description}[format=\textup,format=\textbf]
	
	\item[Discrete Distribution]  produces $\bsX_0, \bsX_1, \dots$ mimicking $\calU[0,1]^d$;
	
	\item[True Measure] $\bst \mapsto \lambda (\bst) \D \bst$  defines the original integral, e.g., Gaussian or Lebesgue;
	
	\item[Integrand] $g$  defines the original, and $f$ defines the transformed version to fit the \texttt{DiscreteDistribution}; and
	
	\item[Stopping Criterion] determines how large $n$ should be to ensure that $\QMCPYabs{\mu - \hmu} \le \varepsilon$ as in (\ref{eq:err_req}).
\end{description}

The software libraries referenced in Section \ref{sec:available} provide one or more of these components. QMCPy combines multiple examples of all these components under an object oriented framework. Each example is implemented as a concrete class that realizes the properties and methods required by the abstract class for that component. The following sections detail descriptions and specific examples for each component. 

Thorough documentation of all QMCPy classes is available in \cite{QMCPyDocs}. Demonstrations of how QMCPy work are given in Google Colab notebooks \cite{QMCPyTutColab2020,QMCPyTutColab2020_paper}. The project may be installed from PyPI into your Python 3 environment via the command \texttt{pip install qmcpy}. In the code snippets that follow, we assume QMCPy has been imported alongside NumPy \cite{numpy} via the following commands in a Python Console:
\lstinputlisting[style=Python]{QMCSoftwareArticle/python/import.txt}

\section{Discrete Distributions}

LD sequences typically mimic $\calU[0,1]^d$.  Good sequences mimicking other distributions are obtained by transformations as described in the next section.  

QMCPy implements \emph{extensible} LD sequences, i.e., those that allow practitioners to obtain and use $\bsX_n, \bsX_{n+1}, \ldots $ without discarding $\bsX_0, \ldots, \bsX_{n-1}$.  Halton sequences do not have preferred sample sizes $n$, but integration lattices and digital sequences prefer $n$ to be a power of $2$.  

These latter two also have an elegant group structure, which we summarize in Table \ref{tab:GroupProp}.  The addition operator  is $\oplus$.  The unshifted sequence is $\bsZ_0, \bsZ_1, \ldots$ and the randomly shifted sequence is $\bsX_0, \bsX_1, \ldots$

\begin{table}[h]
	\centering
	\caption{Properties of lattices and digital net sequences.  Note that they share group properties but also have distinctives.} \label{tab:GroupProp}
\[
	\renewcommand{\arraystretch}{1.3}
\begin{array}{c@{\qquad}c}
	\toprule
	\multicolumn{2}{c}{\text{Define \ldots}} \\
	\multicolumn{2}{c}{\bsZ_1, \bsZ_2, \bsZ_4, \ldots \in [0,1)^d \text{ chosen well} } \\
	\multicolumn{2}{c}{
	\bsZ_{i} := i_0  \bsZ_1 \oplus i_1 \bsZ_{2} \oplus i_2  \bsZ_{4} \oplus  i_3  \bsZ_{8} \oplus \cdots 
	\quad
	\text{for }i = i_0 +i_1 2 + i_2 4 + i_3 8 + \cdots, \; i_\ell \in \{0,1\}} \\
    \multicolumn{2}{c}{\bsX_i := \bsZ_i \oplus \bsDelta, \qquad \text{where }\bsDelta \IIDsim [0,1)^d} \\  \hline
	\text{Rank-1 Integration Lattices} & \text{Digital Nets} \\
		\bst \oplus \bsx : = (\bst + \bsx) \bmod \bsone & \bst \oplus \bsx := \text{binary digitwise addition}, \oplus_{\dig} \\ 
		\text{require } \bsZ_{2^{m}} \oplus \bsZ_{2^{m}} = \bsZ_{\lfloor2^{m-1}\rfloor}
		\quad \forall m \in \bbN_0 
%		 \\ \text{require } \begin{array}{l} \bsZ_1 = (1/2, \ldots, 1/2) \\
%		 	\bsZ_{2^{m}} \oplus \bsZ_{2^{m}} = \bsZ_{2^{m-1}} \quad \forall m \in \bbN 
%	 	\end{array}
		\\
\toprule
\multicolumn{2}{c}{\text{Then it follows that \ldots}} \\
	\multicolumn{2}{c}{\left . \begin{array}{r}
			\calP_m := \{\bsZ_0, \ldots, \bsZ_{2^m-1}\}, \quad
			\bsZ_i \oplus \bsZ_j \in \calP_m \\
			\calP_{\bsDelta,m} := \{\bsX_0, \ldots, \bsX_{2^m-1}\}, \quad
			\bsX_i \oplus \bsX_j \ominus \bsX_k \in \calP_{\bsDelta,m}
	\end{array} \right \}\quad \begin{array}{l}\forall  i,j,k \in \{0, \ldots, 2^{m} -1\} \\ \forall m \in \bbN_0\end{array}} \\
\bottomrule
\end{array}
\]
\end{table}

We illustrate lattice and Sobol' sequences using QMCPy. First, we create an instance of a $d=2$ dimensional \texttt{Lattice} object of the  \texttt{DiscreteDistribution} abstract class. Then we generate the first eight (non-randomized) points in this lattice. 
\lstinputlisting[style=Python]{QMCSoftwareArticle/python/dd_lattice.txt}
The first three generators for this lattice are $\bsZ_1 = (0.5, 0.5)$, $\bsZ_2 = (0.25, 0.75)$, and $\bsZ_4 = (0.125, 0.375)$.  One can check that $(\bsZ_2 + \bsZ_4) \bmod \bsone = (0.375, 0.125) = \bsZ_6$, as Table \ref{tab:GroupProp} specifies.

The random shift has been turned off above to illuminate the group structure.  We normally include the randomization to ensure that there are no points on the boundary of $[0,1]^d$.  Then, when points are transformed to mimic distributions such as the Gaussian, no LD points will be transformed to infinity.  Turning off the randomization generates a warning when the \texttt{gen\_samples} method is called.

Now, we generate Sobol' points using a similar process as we did for lattice points.  Sobol' sequences are one of the most popular example of digital sequences.
\lstinputlisting[style=Python]{QMCSoftwareArticle/python/dd_sobol.txt}
Here, $\bsZ_4$ differs from that for lattices, but more importantly, addition for digital sequences differs from that for lattices.  Using digitwise addition for digital sequences, we can confirm that according to Table \ref{tab:GroupProp},
\begin{align*}
\bsZ_2 \oplus_{\dig} \bsZ_4 = (0.25,0.75)  \oplus_{\dig} (0.125,0.625) \\
=  ({}_20.010,{}_20.110)  \oplus_{\dig} ({}_20.001,{}_20.101) &= ({}_20.011,{}_20.011) \\
& = (0.375,0.375) = \bsZ_6.
\end{align*}

By contrast, if we construct a digital sequence using the generators for the lattice above with $\bsZ_2 = (0.25, 0.75)$, and $\bsZ_4 = (0.125, 0.375)$, we would obtain
\begin{multline*}
\bsZ_6 = \bsZ_2 \oplus_{\dig} \bsZ_4   = ({}_20.010,{}_20.110)  \oplus_{\dig} ({}_20.001,{}_20.011)  \\
= ({}_20.011,{}_20.101) = (0.375, 0.625),
\end{multline*}
which differs from the $\bsZ_6=(0.375, 0.125)$ constructed for lattices.  To emphasize, lattices and digital sequences are different, even if they share the same generators, $\bsZ_1, \bsZ_2, \bsZ_4, \ldots$.

The examples of \texttt{qp.Lattice} and \texttt{qp.Sobol} illustrate how QMCPy LD generators share a common user interface.  The dimension is specified when the instance is constructed, and the number of points is specified when the \texttt{gen\_samples} method is called.  Following Python practice, parameters can be input without specifying their names if they are input in the prescribed order.  QMCPy also includes Halton sequences and IID sequences, again deferring details to the QMCPy documentation \cite{QMCPyDocs}.

A crucial difference between IID generators and LD generators is reflected in the behavior of generating $n$ points.  For an IID generator, asking for $n$ points repeatedly gives you different points each time because they are meant to be random and independent.
\lstinputlisting[style=Python]{QMCSoftwareArticle/python/dd_iidone.txt}
Your output may look different depending on the seed used to generate these random numbers.

On the other hand for an LD generator, asking for $n$ points repeatedly gives you \emph{the same} points each time because they are meant to be the first $n$ points of a specific LD sequence.  
\lstinputlisting[style=Python]{QMCSoftwareArticle/python/dd_latticeone.txt}
Here we allow the randomization so that the first point in the sequence is not the origin.  To obtain the \emph{next} $n$ points one may specify the start and ending indices of the sequence.
\lstinputlisting[style=Python]{QMCSoftwareArticle/python/dd_latticenextone.txt}

Figure \ref{fig:increase_n} shows how increasing the number of lattice and Sobol' LD points through powers of two fills in the gaps in an even way.

\begin{figure}[t]
	\includegraphics[width=1\textwidth]{QMCSoftwareArticle/figs/dd_lattice_successive.png}
	\qquad
	\includegraphics[width=1\textwidth]{QMCSoftwareArticle/figs/dd_sobol_successive.png}
	\caption{Randomized lattice and Sobol' points mimicking a $\calU[0,1]^2$ measure for $n = 64, 128,$ and 256. Note how increasing the number of points evenly fills in the gaps between points.}
	\label{fig:increase_n}
\end{figure}

\section{True Measures}

The LD sequences implemented as \texttt{DiscreteDistribution} objects mimic the $\calU[0,1]^d$ distribution.  However, we may need sequences to mimic other distributions.  This is implemented via variable transformations, $\bsPsi$.  In general, if $\bsX \Msim \calU[0,1]^d$, then
\begin{subequations} \label{eq:exampleVarTrans}
\begin{gather}
\bsT = \bsPsi(\bsX) := \bsa  + (\bsb - \bsa) \odot \bsX \Msim  \calU[\bsa,\bsb], \\
\label{eq:exampleVarTransGauss}
\bsT = \bsPsi(\bsX) := \bsa + \mA \bsPhi^{-1}(\bsX)  \Msim \calN(\bsa, \mSigma), \\
\nonumber  \text{where }  \bsPhi^{-1}(\bsX) : = \begin{pmatrix} \Phi^{-1}(X_1) \\ \vdots \\ \Phi^{-1}(X_d)\end{pmatrix}, \qquad \mSigma = \mA \mA^T,
\end{gather}
\end{subequations}
and $\odot$ denotes term-by-term (Hadamard) multiplication.  Here, $\bsa$ and $\bsb$ are assumed to be finite, and $\Phi$ is the standard Gaussian distribution function.  Again we use $\Msim$ to denote mimicry, not necessarily in a probabilistic sense.

Figure \ref{fig:tm_ug} displays LD sequences transformed as described above to mimic a uniform and a Gaussian distribution.  The code to generate these points takes the following form for uniform points based on a Halton sequence: 
\lstinputlisting[style=Python]{QMCSoftwareArticle/python/tm_uniform.txt}
whereas for Gaussian points based on a lattice sequence, we have:
\lstinputlisting[style=Python]{QMCSoftwareArticle/python/tm_gaussian.txt}
Here the covariance decomposition $\mSigma = \mA \mA^T$ is done using principal component analysis. The Cholesky decomposition is also available.

\begin{figure}[t]
	\includegraphics[width=.45\textwidth]{QMCSoftwareArticle/figs/tm_uniform.png} 
	%	\quad
	\includegraphics[width=.45\textwidth]{QMCSoftwareArticle/figs/tm_gaussian.png}
	\caption{Sobol' samples transformed to mimic a uniform $\calU\left(\begin{bmatrix} -2 \\ 0 \end{bmatrix}, \begin{bmatrix} 2 \\ 4 \end{bmatrix} \right)$ (left) and  Gaussian $\calN\left(\begin{bmatrix} 3 \\ 2 \end{bmatrix}, \begin{bmatrix} 9 & 5 \\ 5 & 4 \end{bmatrix} \right)$ (right) distribution.}
	\label{fig:tm_ug}
\end{figure}

The Brownian motion distribution arises often in financial risk applications.  Here the $d$ components of the variable $\bsT$ correspond to the discretized Brownian motion at times $\tau/d, 2\tau/d, \ldots, \tau$, where $\tau$ is the time horizon.  The distribution is a special case of the Gaussian with covariance 
\begin{equation} \label{eq:BMcov}
	\mSigma = (\tau/d) \bigl (\min(j,k) \bigr)_{j,k=1}^d
\end{equation}
and mean $\bsa$, which  corresponds to a drift coefficient times $(\tau/d)(1, 2, \ldots, d)^T$. The code for generating a Brownian motion is
\lstinputlisting[style=Python]{QMCSoftwareArticle/python/tm_brownian_motion.txt}
Figure \ref{fig:tm_bm} displays a Brownian motion based on Sobol' sequence with and without a drift.

\begin{figure}[t]
	\includegraphics[width=1\textwidth]{QMCSoftwareArticle/figs/tm_bm.png} 
	\caption{Sobol' samples transformed to mimic a 52-dimensional Brownian Motion without drift (left) and with drift coefficient 2 (right).}
	\label{fig:tm_bm}
\end{figure}

\section{Integrands}

Let's return  to the integration problem in \eqref{eq:integral}, which we must rewrite as \eqref{eq:fintegral}.  We choose a transformation of variables defined as $\bst = \bsPsi(\bsx)$ where $\bsPsi:\cube \to \calT$.  This leads to 
\begin{align}
	\nonumber 
 \mu &= \int_\calT g(\bst) \, \lambda(\bst) \, \D \bst  = \int_{\cube} g\bigl(\bsPsi(\bsx)\bigr) \, \lambda\bigl(\bsPsi(\bsx)\bigr) \,\QMCPYabs{\bsPsi'(\bsx)}  \, \D \bsx =  \int_{\cube}f(\bsx) \, \D \bsx,  \\
 \label{eq:transVar}
  & \qquad \qquad \text{where } f(\bsx)  = g\bigl(\bsPsi(\bsx)\bigr)  \, \lambda\bigl(\bsPsi(\bsx)\bigr) \,\QMCPYabs{\bsPsi'(\bsx)} ,
\end{align}
and $\QMCPYabs{\bsPsi'(\bsx)} := \QMCPYabs{\partial \bsPsi/\partial \bsx}$ represents the Jacobian of the variable transformation.  The abstract class \texttt{Integrand} provides $f$ based on the user's input of $g$ and the \texttt{TrueMeasure} instance, which defines $\lambda$ and the transformation $\bsPsi$. Different choices of $\bsPsi$ lead to different $f$, which may give different rates of convergence of the cubature, $\hmu$ to $\mu$.

We illustrate the \texttt{Integrand} class via an example of Keister \cite{Kei96}:
\begin{equation} \label{eq:KeisterIntegral}
	\mu 
	= \int_{\bbR^d} \cos(\lVert \bst \rVert) \exp(-\bst^T \bst) \, \D \bst \\ 
	= \int_{\bbR^d} \underbrace{\pi^{d/2} \cos(\lVert \bst \rVert)}_{g(\bst)}\, \underbrace{\pi^{-d/2} \exp(-\bst^T \bst) }_{\lambda(\bst)} \, \D \bst.
\end{equation}
Since $\lambda$ is the density for $\calN(\bszero,\mI/2)$, it is natural to choose $\bsPsi$ according to \eqref{eq:exampleVarTransGauss} with $\mA = \sqrt{1/2} \, \mI$, in which case $\lambda(\bsPsi(\bsx)) \QMCPYabs{\bsPsi'(\bsx)}  = 1$, and so 
\[
\mu = \int_{[0,1]^d} \underbrace{\pi^{d/2} \cos(\lVert \bsPsi(\bsx) \rVert)}_{f(\bsx)} \, \D \bsx, \qquad 
\bsPsi(\bsx) := \sqrt{1/2} \,\bsPhi^{-1}(\bsx).
\]

The code below sets up an \texttt{Integrand} instance using QMCPy's \texttt{CustomFun} wrapper to tie a user-defined function $g$ into the QMCPy framework.  Then we evaluate the sample mean of $n=1000$ $f$ values obtained by sampling at transformed Halton points.
\lstinputlisting[style=Python]{QMCSoftwareArticle/python/i_keister.txt}
We have no indication yet of how accurate our approximation is.  That topic is treated in the next section.  Figure \ref{fig:ikc} visualizes sampling on the original integrand, $g$, and sampling on the transformed integrand, $f$. 

\begin{figure}[t]
	\includegraphics[height=6cm]{QMCSoftwareArticle/figs/i_keister_contours.png}
	\caption{Right: Sampling the transformed integrand $f$ at Halton points $\bsX_i \LDsim \calU[0,1]^2$. Left: Sampling the original integrand $g$ at $\bsT_i = \bsPsi(\bsX_i) \Msim \calN(\bszero,\mI/2)$ where $\bsPsi$ is defined in \eqref{eq:exampleVarTransGauss}.  } \label{fig:ikc}
\end{figure}

Another way to approximate the Keister integral in \eqref{eq:KeisterIntegral} is to write it as an integral with respect to the Lebesgue measure:
\begin{align*} 
	\mu 
	& = \int_{\bbR^d} \underbrace{\cos(\lVert \bst \rVert) \exp(-\bst^T \bst) }_{g(\bst)} \, \underbrace{1}_{\lambda(\bst)} \,\D \bst \\
	& = \int_{[0,1]^d} \underbrace{\cos(\lVert \bsPsi(\bsx) \rVert) \exp(-\bsPsi^T\!\!(\bsx) \bsPsi(\bsx)) \QMCPYabs{\bsPsi'(\bsx)}  }_{f(\bsx)} \, \D \bsx,
\end{align*}
where $\bsPsi$ is any transformation from $[0,1]^d$ to $\bbR^d$. Now $\lambda$ is \emph{not} a PDF.  QMCPy can perform the cubature this way as well.
\lstinputlisting[style=Python]{QMCSoftwareArticle/python/i_keisterLebesgue.txt}
The $\bsPsi$ chosen when transforming uniform sequences on the unit cube to fill $\bbR^d$ is  given by \eqref{eq:exampleVarTransGauss} with $\mA = \mI$.

In the examples above, one must input the correct $g$ into \texttt{CustomFun} along with the correct \texttt{TrueMeasure} $\lambda$ to define the integration problem. The \texttt{Keister} integrand included in the QMCPy library takes a more flexible approach of defining the integration problem $\mu$ in \eqref{eq:KeisterIntegral}. Selecting a different \texttt{sampler} $\bsPsi$ performs  \emph{importance sampling} which leaves $\mu$ unchanged.  
\lstinputlisting[style=Python]{QMCSoftwareArticle/python/i_keisterBuiltIn.txt}

In the first case above, the $\lambda$ in \eqref{eq:transVar} corresponds to the Gaussian density  with mean zero and variance $1/2$ by default, and the corresponding variable transformation, $\bsPsi$, is chosen to make $\lambda\bigl(\bsPsi(\bsx)\bigr) \QMCPYabs{\bsPsi'(\bsx)}  = 1$ and $f(\bsx) = g\left(\bsPsi(\bsx) \right)$.  In the second case,  we choose an importance sampling density $\lambda_{\IMP}$, corresponding  to standard  Gaussian, and the variable transformation $\bsPsi_{\IMP}$ makes $\lambda_\IMP\bigl(\bsPsi_{\IMP}(\bsx)\bigr) \QMCPYabs{\bsPsi'_{\IMP}(\bsx)}  = 1$.  Then
\begin{align}
	\nonumber 
	\mu &= \int_\calT g(\bst) \, \lambda(\bst) \, \D \bst  = \int_\calT g(\bst) \, \frac{\lambda(\bst)}{\lambda_\IMP(\bst) } \, \lambda_\IMP(\bst) \D \bst  \\ 
	\nonumber
	& =  \int_{\cube} g\bigl(\bsPsi_\IMP(\bsx)\bigr) \, \frac{\lambda\bigl(\bsPsi_\IMP(\bsx)\bigr)} {\lambda_\IMP\bigl(\bsPsi_\IMP(\bsx)\bigr) } \lambda_\IMP\bigl(\bsPsi_\IMP(\bsx)\bigr) \,\QMCPYabs{\bsPsi'_{\IMP}(\bsx)}  \, \D \bsx \\
	\nonumber
	& =  \int_{\cube} f_\IMP(\bsx) \, \D \bsx  \\
	\label{eq:transVarImp}
	& \qquad \qquad \text{where } f_\IMP(\bsx)  = g\bigl(\bsPsi_\IMP(\bsx)\bigr)  \,  \frac{\lambda\bigl(\bsPsi_\IMP(\bsx)\bigr)} {\lambda_\IMP\bigl(\bsPsi_\IMP(\bsx)\bigr) }.
\end{align}
Because LD samples mimic $\calU\cube$, choosing a different \texttt{sampler} is equivalent to choosing a different variable transform.

\section{Stopping Criteria} \label{sec:stopping_crit}

The \texttt{StoppingCriterion} object determines the number of samples $n$ that are required for the sample mean approximation $\hmu$ to be within error tolerance $\varepsilon$ of the true mean $\mu$.  Several QMC stopping criteria have been implemented in QMCPy, including replications, stopping criteria that track the decay of the Fourier complex exponential or Walsh coefficients of the integrand \cite{HicJim16a,HicEtal17a,JimHic16a}, and stopping criteria based on Bayesian credible intervals \cite{RatHic19a,JagHic22a}.

Let us return to the Keister example from the previous section.  After setting up  a default \texttt{Keister} instance via a Sobol' \texttt{DiscreteDistribution}, we choose a \texttt{StoppingCriterion} object that matches the \texttt{DiscreteDistribution} and input our desired tolerance.  Calling the  \texttt{integrate} method returns the approximate integral plus some useful information about the computation.
\lstinputlisting[style=Python]{QMCSoftwareArticle/python/sc_keister_qmc.txt}
The second output of the stopping criterion provides helpful diagnostic information.  This computation requires $n=2^{13}$ Sobol' points and $0.009$ seconds to complete.  The error bound is $0.000637$, which falls below the absolute tolerance.

QMC, which uses LD sequences, is touted as providing substantially greater computational efficiency compared to IID MC.
Figure \ref{fig:sc_comp} compares the time and sample sizes needed to compute the $5$-dimensional Keister integral \eqref{eq:KeisterIntegral} using IID sequences and LD lattice sequences. Consistent with what is stated in Section \ref{sec:intro}, the error of IID MC is $\calO(n^{-1/2})$, which means that the time and sample size to obtain an absolute error tolerance of $\varepsilon$ is $\calO(\varepsilon^{-2})$.  By contrast, the  error of QMC using LD sequences is $\calO(n^{-1+\epsilon})$, which implies $\calO(\varepsilon^{-1-\epsilon})$ times and sample sizes.  We see that QMC methods often require orders of magnitude fewer samples than MC methods to achieve the same error tolerance.

\begin{figure}[t]
	\includegraphics[height=6cm]{QMCSoftwareArticle/figs/sc_comp.png}
	\caption{Comparison of run times and sample sizes for computing the $5$-dimensional Keister integral \eqref{eq:KeisterIntegral} using IID and LD lattice sequences for a variety of absolute error tolerances. The respective stopping criteria are  \texttt{qp.CubMCG} \cite{HicEtal14a} and  \texttt{qp.CubQMCLatticeG} \cite{JimHic16a}. The LD sequences provide the desired answer much more efficiently.}
	\label{fig:sc_comp}
\end{figure}

For another illustration of QMC cubature, we turn to pricing an Asian arithmetic mean call option. The payoff of this option is the positive difference between the strike price, $K$, averaged over the time horizon: 
$$
\text{payoff}(\bsS) = \max\left(\frac{1}{2d}\sum_{j=1}^d (S_{j-1}+S_j)-K, \; 0\right), \quad \bsS = (S_0, \ldots, S_d).
$$
Here $S_j$ denotes the asset price at time $\tau j/d$, and a trapezoidal rule is used for discrete approximation of the integral in time that defines the average.  A basic model for asset prices is a geometric Brownian motion, 
\[
S_j(\bsT) = S_0 \exp((r - \sigma^2/2) \tau j/d + \sigma T_j),   \;  j = 1, \ldots, d, \; \bsT = (T_1, \ldots, T_d)\sim \calN(\bszero,\mSigma),
\]
where $\mSigma$ is defined in \eqref{eq:BMcov}, $r$ is the interest rate, $\sigma$ is the volatility, and $S_0$ is the initial asset price.  The fair price of the option is then the expected value of the discounted payoff, namely,
\begin{equation*}
	\text{price} = \mu = \bbE[g(\bsT)], \quad \text{where } g(\bst) = \text{payoff}\bigl(\bsS(\bst) \bigr) \exp(-r \tau).
\end{equation*}

The following code utilizes QMCPy's Asian option \texttt{Integrand} object to approximate the Asian call option for a particular choice of the parameters.
\lstinputlisting[style=Python]{QMCSoftwareArticle/python/sc_aco.txt}
Because this \texttt{Integrand} object has the built-in Brownian motion \texttt{TrueMeasure}, one only need provide the LD sampler.

Out of the money option price calculations can be sped up by adding an upward drift to the Brownian motion.  The upward drift produces more in the money paths and also reduces the variation or variance of the final integrand, $f$.  This is a form of importance sampling.  Using a Brownian motion without drift we get
\lstinputlisting[style=Python]{QMCSoftwareArticle/python/wo_imp_samp_aco.txt}
Adding the upward drift gives us the answer faster:
\lstinputlisting[style=Python]{QMCSoftwareArticle/python/w_imp_samp_aco.txt}
The choice of a good drift is an art.  

The improvement in time is less than improvement in $n$ because the integrand is more expensive to compute when the drift is employed.  Referring to \eqref{eq:transVarImp}, in the case of no drift, the $\lambda$ corresponds to the density for the discrete Brownian motion, and the variable transformation $\bsPsi$ is chosen so that $f(\bsx) = g\left(\bsPsi(\bsx) \right)$.  However, in the case of a drift, the integrand becomes  $f_\IMP(\bsx)  = g\bigl(\bsPsi_\IMP(\bsx)\bigr)  \lambda\bigl(\bsPsi_\IMP(\bsx)\bigr)/\lambda_\IMP\bigl(\bsPsi_\IMP(\bsx)\bigr)$, which requires more computation time per integrand value. 

\section{Under the Hood}

In this section, we look at the inner workings of QMCPy and point out features we hope will benefit the community of QMC researchers and practitioners. We also highlight important idiosyncrasies of QMC methods and how QMCPy addresses these challenges. For details, readers should refer to the QMCPy documentation \cite{QMCPyDocs}.

\subsection{LD Sequences}

LD sequences are the backbone of QMC methods. QMCPy provides generators that combine research from across the QMC community to enable advanced features and customization options.

Two popular LD sequences are integration lattices and digital nets which we previously outlined in Table \ref{tab:GroupProp}. These LD generators are comprised of two parts: the static generating vectors $\bsZ_1,\bsZ_2,\bsZ_4, \ldots \in [0,1)^d$ and the callable generator function. By default, QMCPy provides a number of high quality generating vectors for users to choose from. For instance, the default ordinary lattice vector was constructed by Cools, Kuo, and Nuyens \cite{doi:10.1137/06065074X} to support 3600 dimensions and $2^{20}$ samples. However, users who require more samples but fewer dimensions may switch to a generating vector constructed using LatNet Builder \cite{LEcEtal22a,LatNet} to support 750 dimensions and $2^{24}$ samples. Moreover, the \texttt{qp.Lattice} and \texttt{qp.DigitalNet} objects allow users to input their own generating vectors to produce highly customized sequences. To find such vectors, we recommend using LatNet Builder's construction routines as the results can be easily parsed into a QMCPy-compatible format. 

Along with the selection of a generating vector, QMCPy's low discrepancy sequence routines expose a number of other customization parameters. For instance, the lattice generator extends the Magic Point Shop \cite{Nuy17a} to support either linear or natural ordering. Digital sequences permit either standard or Graycode ordering and maybe randomized via a digital shift optionally combined with a linear scrambling \cite{Mat98}. Halton sequences may be randomized via the routines of either Owen \cite{Owe20a} or Hofert and Lemieux \cite{QRNG2020}. 

\subsection{A Word of Caution when Using LD Sequences}

Although QMCPy's \texttt{DiscreteDistribution}s have many of the same parameters and methods, users should be careful when swapping IID sequences with LD sequences. While IID node-sets have no preferred sample size, LD sequences often require special sampling ranges to ensure optimal discrepancy. As mentioned  earlier, digital nets and integration lattices show better evenness for sample sizes that are powers of 2. On the other hand, the preferred sample sizes for $d$-dimensional Halton sequences are $n = \prod_{j=1}^d p^{m_j}_j$ where $p_j$ is the $j^{\text{th}}$ prime number and $m_j \in \mathbb{N}_0$ for $j=1,\dots,d$. Due to the infrequency of such values, Halton sequences are often regarded as not having a preferred sample size.  

Users may also run into trouble when trying to generate too many points. Since QMCPy's generators construct sequences in 32-bit precision, generating greater than $2^{32}$ consecutive samples will cause the sequence to repeat. In the future, we plan to expand our generators to support optional 64-bit precision at the cost of greater computational overhead.

Another subtlety arises when transforming LD sequences to mimic different distributions. As mentioned earlier, unrandomized lattice and digital sequences include the origin, making transformations such as \eqref{eq:exampleVarTransGauss} produce infinite values.  

Some popular implementations of LD sequences drop the first point, which is the origin in the absence of randomization.  The rationale is to avoid the transformation of the origin to infinity when mimicking a Gaussian or other distribution with an infinite sample space.  Unfortunately, dropping the first point destroys some of the nice properties of the first $n = 2^m$ points of LD sequences, which can degrade the order of convergence for QMC cubature. A careful  discussion of this matter is given by~\cite{owen2020dropping}.

\subsection{Transformations}

The transformation $\bsPsi$  connects a \texttt{DiscreteDistribution} and \texttt{TrueMeasure}. So far, we have assumed the \texttt{DiscreteDistribution} mimics a $\calU[0,1]^d$ distribution with PDF $\varrho(\bsx)=1$. However, it may be advantageous to utilize a \texttt{DiscreteDistribution} that mimics a different distribution. For instance, QMCPy includes a \texttt{DiscreteDistribution} corresponding to an IID standard Gaussian sequence.  It may be transformed to mimic a general Gaussian distribution \texttt{TrueMeasure} while avoiding the inverse CDF transform required for nodes mimicking $\calU[0,1]^d$. 

Suppose we have a \texttt{DiscreteDistribution} mimicking  density $\varrho$ supported on $\calX$.  Then using the variable transformation $\bsPsi$, 
\begin{equation*}
	\mu = \int_\calT g(\bst) \, \lambda(\bst) \, \D \bst  =  \int_{\calX}f(\bsx) \, \varrho(\bsx) \D \bsx \quad \text{for }
   f(\bsx)  = g\bigl(\bsPsi(\bsx)\bigr)  \, \frac{\lambda\bigl(\bsPsi(\bsx)\bigr)}{\varrho(\bsx)} \, \lvert \bsPsi'(\bsx)\rvert,
\end{equation*}
which generalizes \eqref{eq:transVar}.
QMCPy also includes support for composed variable transformations. Suppose that the variable transformation is a composition of several transformations: $\bsPsi=\widehat{\bsPsi}_L = \bsPsi_L \circ \bsPsi_{L-1} \circ \dots \circ \bsPsi_1$ as in \eqref{eq:transVarImp}. Here, $\bsPsi_l:\calX_{l-1} \to \calX_l$,  $\calX_0=\calX$, and $\calX_L=\calT$ so that the transformations are compatible with the \texttt{DiscreteDistribution} and \texttt{TrueMeasure}. Let $\widehat{\bsPsi}_l=\bsPsi_l \circ \bsPsi_{l-1} \circ \dots \circ \bsPsi_1$ denote the composition of the first $l$ transforms and assume that $\widehat{\bsPsi}_0(\bsx)=\bsx$, the identity transform. Then we may write $\mu =  \int_{\calX}f(\bsx) \, \varrho(\bsx) \D \bsx$ for 
\begin{equation*}
    f(\bsx) 
 = g\bigl(\widehat{\bsPsi}_L(\bsx)\bigr)\frac{\lambda\bigl(\widehat{\bsPsi}_L(\bsx)\bigr)}{\varrho(\bsx)}\prod_{l=1}^L \bigl\lvert \bsPsi_l'\bigl(\widehat{\bsPsi}_{l-1}(\bsx)\bigr) \bigr\rvert .
\end{equation*}
It is often the case that $\bsPsi_l$ is chosen such that $\bsPsi_l(\bsX)$ is stochastically equivalent to a random variable with density $\lambda_l$ on sample space $\calX_l$ when $\bsX$ is a random variable with density $\varrho_l$ on sample space $\calX_{l-1}$.  This implies  $\varrho_l(\bsx) = \lambda_l(\bsPsi_l(\bsx))\lvert  \bsPsi'_l (\bsx) \rvert$ so that
\begin{equation*}
f(\bsx) 
 = g\bigl(\widehat{\bsPsi}_L(\bsx)\bigr)\frac{\lambda\bigl(\widehat{\bsPsi}_L(\bsx)\bigr)}{\varrho(\bsx)}\prod_{l=1}^L \frac{\varrho_l(\widehat{\bsPsi}_{l-1}(\bsx))}{\lambda_l(\widehat{\bsPsi}_l(\bsx))} .
\end{equation*}

For an example, we return to the Keister integral \eqref{eq:KeisterIntegral}. The following code constructs three \texttt{Keister} instances: one without importance sampling, one importance sampled by a Gaussian distribution, and one importance sampled by the composition of a Gaussian distribution with a Kumaraswamy distribution. All \texttt{Integrand}s use a Sobol' \texttt{DiscreteDistribution}, making $\varrho(\bsx)=1$ and $\calX = [0,1]^d$. The \texttt{TrueMeasure} is $\calN(\bszero,\mI/2)$ making $\lambda(\bst) = \pi^{-d/2} \exp(-\bst^T \bst)$ and $\calT = \bbR^d$. 

The table below displays the variable transformations and the measures for these three cases.  In all cases $\varrho_1(\bsx) = \cdots = \varrho_L(\bsx) = 1$ because the $\bsPsi_l$ utilize inverse cumulative distributions.

\[\begin{array}{ccccccc}
	\texttt{Integrand} & L & \lambda_1 & \bsPsi_1 & \lambda_2 & \bsPsi_2 & f \tabularnewline
	\toprule
	\texttt{K} & 1 & \calN(\bszero, \mI/2) & \eqref{eq:exampleVarTransGauss}& && g(\bsPsi_1(\cdot)) \tabularnewline [1ex]
	\texttt{K\_gauss} & 1 & \calN(\bszero,3 \mI/4) & \eqref{eq:exampleVarTransGauss} & && 
	\displaystyle g(\bsPsi_1(\cdot)) \frac{\lambda(\bsPsi_1(\cdot))}{\lambda_1(\bsPsi_1(\cdot))} \tabularnewline [2.5ex]
	\texttt{K\_gauss\_kuma} & 2 & \Kum & \bsF^{-1}_\Kum & \calN(\bszero,\mI) &  \eqref{eq:exampleVarTransGauss} &
	\displaystyle \frac{g(\bsPsi_2(\bsPsi_1(\cdot))) \lambda(\bsPsi_2(\bsPsi_1(\cdot))) }{\lambda_1(\bsPsi_1(\cdot)) \lambda_2(\bsPsi_2(\bsPsi_1(\cdot)))} 
\end{array}
\]

Here $\Kum$ denotes the multivariate Kumaraswamy distribution with independent marginals, and $ \boldsymbol{F}^{-1}_\Kum$ denotes the element-wise inverse cumulative distribution function.  The following code evaluates the Keister integral \eqref{eq:KeisterIntegral} for $d=1$ and error tolerance $\varepsilon = 5\text{E}-8$.  The timings for each of these different integrands are displayed.

\lstinputlisting[style=Python]{QMCSoftwareArticle/python/composed_IS.txt}

Successful importance sampling places more points where the original integrand, $g$, varies more.  Equivalently, successful importance sampling makes the transformed integrand, $f$, flatter. The shorter execution times correspond to flatter integrands, as  illustrated in Figure \ref{fig:mIS}. The above example uses $d=1$ to facilitate the plot in Figure \ref{fig:mIS}, however, the same example works for arbitrary dimensions.
\begin{figure}[t]
    \centering
	\includegraphics[width=.8\textwidth]{QMCSoftwareArticle/figs/multiple_IS.png}
	\caption{Keister functions with and without importance sampling (\texttt{IS}). Note that the Keister functions using importance sampling are generally less variable and therefore easier to integrate, as evidenced by the faster integration times.} \label{fig:mIS}
\end{figure}

Automatic Bayesian stopping criteria choose $n$ needed to satisfy the error tolerance via a Bayesian credible interval for $\mu$ \cite{RatHic19a,JagHic22a}.  These are also options in QMCPy.


\section{Further Work} \label{sec:further}

QMCPy is ripe for growth and development in several areas.  We hope that the QMC community will join us in making this a reality.

Multi-level (quasi-)Monte Carlo (ML(Q)MC) methods make possible the computation of expectations of functionals of stochastic differential equations and partial differential equations with random coefficients.  Such use cases appear in quantitative finance and in geophysical applications.  QMCPy's ML(Q)MC's capability is rudimentary, but under active development.

We hope to add a greater variety of use cases and are engaging collaborators to help.  Sobol' indices, partial differential equations with random coefficients, expected improvement measures for Bayesian optimization, and multivariate probabilities are some of those on our radar.

Recently, some QMC experts have focused on developing LD generators for Python. Well established packages such as SciPy \cite{SCIPY} and PyTorch \cite{PyTorch} are actively working to develop QMC modules that support numerous LD sequences and related functionalities. When these modules are released to the public, we plan to integrate the routines as optional backends for QMCPy's LD generator objects. Creating ties to these other packages will allow users to call their preferred generators from within the QMCPy framework.  Moreover, as features in QMCPy become more common and prove their value, we will try to incorporate them into SciPy and other popular, general purpose packages.

We also plan to expand our library of digital net generating matrices. We wish to incorporate interlaced digital nets, polynomial lattices, and Niederreiter sequences, among others. By including high quality defaults in QMCPy, we hope to make these sequences more readily available to the public. 

Our \texttt{DiscreteDistrution} places equal weights on each support point, $\bsX_i$.  In the future we might generalize this to unequal weights.

QMCPy already includes importance sampling, but the choice of sampling distribution must be chosen a priori.  We would like to see an automatic, adaptive choice.

Control variates can be useful for QMC, as well as for IID MC \cite{HicEtal03}.  These should be incorporated into QMCPy in a seamless way.

We close with an invitation.  Try QMCPy.  If you find bugs or missing features, submit an issue to \href{https://github.com/QMCSoftware/QMCSoftware/issues}{\nolinkurl{https://github.com/QMCSoftware/QMCSoftware/issues}}.  If you wish to add your great algorithm or use case, submit a pull request to our GitHub repository at \href{https://github.com/QMCSoftware/QMCSoftware/pulls}{\nolinkurl{https://github.com/QMCSoftware/QMCSoftware/pulls}}.  We hope that the community will embrace QMCPy.

\begin{acknowledgement}
The authors would like to thank the organizers for a wonderful MCQMC 2020. 
We also thank the referees for their many helpful suggestions.  This work is supported in part by SigOpt and National Science Foundation grant DMS-1522687.
\end{acknowledgement}

%\section*{References}
%\nocite{*}
\bibliographystyle{spmpsci.bst}
\bibliography{FJH23,FJHown23,QMCSoftwareArticle}


\iffalse
\AGSComment{

\section{Extra Derivations}

\subsection{Importance sampling with Gaussian}

$$ \bsPsi(\bsx) = A\Phi^{-1}(\bsx) + \mu \quad \text{where} \quad \Sigma = AA^T$$
$$ \left\lvert \frac{\D \bsPsi}{\D \bsx} \right\rvert = \det(J) \quad \text{for} \quad J_{ij} = \frac{\D \Psi_i}{\D x_j} = \frac{A_{ij}}{\phi(\Phi^{-1}(x_j))}$$
$$ \det(J) = \det\left(A . \text{diag}\left(\frac{1}{\phi(\Phi^{-1}(\bsx))}\right)\right) = \frac{\det(A)}{\prod_{j=1}^d \phi(\Phi^{-1}(x_j))}$$}

\FJHComment{Given $g$ and $\lambda$ to define your integral, suppose that $\lambda(\bst)$ is some multiple of some multivariate normal distribution:
	\begin{equation*}
		\lambda(\bst)  = \frac{c \exp\bigl(-(\bst -\bsb)^T \mLambda^{-1}(\bst -\bsb)/2\bigr)}{\sqrt{ (2\pi)^d \det(\mLambda)}} 
	\end{equation*}
Suppose that we use the variable transformation $\bsPsi$ defined in \eqref{eq:exampleVarTransGauss}.  By what you have derived above:
\begin{align*}
	\QMCPYabs{\frac{\partial \bsPsi}{\partial \bsx}} 
	& = \frac{\det(\mA^T)}{\prod_{j=1}^d \phi(\Phi^{-1}(x_j))} = \sqrt{(2\pi)^d \det(\mSigma)} \exp\bigl( \QMCPYnormnorm[2]{\bsPhi^{-1}(\bsx)}^2/2\bigr) \\
	\MoveEqLeft[2]{\lambda\bigl(\bsPsi(\bsx)\bigr) \QMCPYabs{\frac{\partial \bsPsi}{\partial \bsx}}} \\
	& =  c \sqrt{\frac{\det(\mSigma)}{\det(\mLambda)}} \exp \bigl( [ \bsPhi^{-1}(\bsx)^T \bsPhi^{-1}(\bsx) \\
	& \qquad \qquad -   (\mA\bsPhi^{-1}(\bsx) +\bsa -\bsb)^T\mLambda^{-1}(\mA\bsPhi^{-1}(\bsx) +\bsa -\bsb)        ] /2   \bigr) \\
	& =  c \sqrt{\frac{\det(\mSigma)}{\det(\mLambda)}} \exp \bigl( [ \bsPhi^{-1}(\bsx)^T \bsPhi^{-1}(\bsx) 
	-   \bsPhi^{-1}(\bsx)^T\mA^T\mLambda^{-1}\mA \bsPhi^{-1}(\bsx)\\
	& \qquad \qquad  - 2(\bsa -\bsb)^T \mLambda^{-1}\mA\bsPhi^{-1}(\bsx) - (\bsa -\bsb)^T \mLambda^{-1} (\bsa -\bsb)] /2   \bigr) 
\end{align*}
If $\mLambda = \mSigma$, then this simplifies to
\begin{equation*}
		\lambda\bigl(\bsPsi(\bsx)\bigr) \QMCPYabs{\frac{\partial \bsPsi}{\partial \bsx}}
	 =  c \exp \bigl( - (\bsa -\bsb)^T \mA^{-T} [
	2 \bsPhi^{-1}(\bsx) + \mA^{-1}(\bsa -\bsb)] /2   \bigr) .
\end{equation*}
If $\mLambda = \lambda^2 \mI$ and  $\mA = \sigma \mI$, then 
\begin{align*}
	\lambda\bigl(\bsPsi(\bsx)\bigr) \QMCPYabs{\frac{\partial \bsPsi}{\partial \bsx}}
	& = \frac{c\sigma}{\lambda} 
	\exp \bigl( [ (\lambda^2 -\sigma^2)\bsPhi^{-1}(\bsx)^T\bsPhi^{-1}(\bsx) 
	\\
	& -  (\bsa -\bsb)^T [
	2\sigma \bsPhi^{-1}(\bsx) + (\bsa -\bsb)] /(2\lambda^2)   \bigr) .
\end{align*}
}

\AGSComment{

\subsection{Chaining Transforms}
With $k$ transforms, let $\widehat{\bsPsi}_j(\bsx) = (\bsPsi_j \circ \bsPsi_{j-1} \circ \dots \circ \bsPsi_1)(\bsx)$ denote the first $j$ compositions and $\bst = \widehat{\bsPsi}_k(\bsx)$ be the complete transformation. Then, 
$$f(\bsx) = g(\widehat{\bsPsi}_k(\bsx)) \frac{\lambda(\widehat{\bsPsi}_k(\bsx))}{\varrho(\bsx)} \prod_{i=1}^{k} \left\lvert \frac{\D \, \bsPsi_i}{\D \, \widehat{\bsPsi}_{i-1}(\bsx)} \right\rvert$$
where $\widehat{\bsPsi}_0(\bsx) = \bsx$.
}

\FJHComment{

\subsection{Software Framework}

\newcommand{\dd}{\texttt{DiscreteDistribution}\xspace}
\newcommand{\tm}{\texttt{TrueMeasure}\xspace}

\begin{itemize}
\item \dd $\rightarrow$ \tm gives an instance that can generate a sequence that mimics the measure specified by \tm if \tm is a probability measure; it also has a \texttt{gen\_samples} method.

\item if \tm is not a probability measure, then \dd $\rightarrow$ \tm sets up a transformation from $\calX$ to $\calT$.

\item \texttt{CustomFun} takes a user-defined function $g$ and a \tm instance and uses the $\lambda$ inferred from the \tm to form $f$.

\item Built-in \texttt{Integrand} classes know what $g$ and $f$ are.

\begin{itemize}
    \item One way is to have these \texttt{Integrand} classes assume a \tm and only need a \dd input. If you do not like the default \tm, then you need to input your own.
    
    \item Another way is to have these \texttt{Integrand} classes take a \tm input and use that, plus their knowledge of $g$ and $\lambda$ to find $f$, perhaps by what could be considered importance sampling.
    
\end{itemize}
\end{itemize}
}

\AGSComment{
\subsection{\texttt{print}ing  QMCPy Instances}

QMCPy provides the a \texttt{print} command to display information about instances of QMCPy classes. In the sample code in Section \ref{sec:stopping_crit} this command was used show the output of a \texttt{StoppingCriteroion}'s \texttt{integrate} method.

faulted for any given component. This feature was already used to print the data object returned by the stopping criterion's \texttt{integrate} method in Section \ref{sec:stopping_crit}. The following code prints a drifted \texttt{BrownianMotion} instance and displays the connection to an underlying Gaussian measure. 
\lstinputlisting[style=Python]{QMCSoftwareArticle/python/print_bm.txt}}

\AGSComment{Do we want the implementation to follow the first or second equation? QMCPy current follows first equation (directly computes Jacobians), which may allow for transforms that do not correspond to PDFs. However, the second equation would be easier for QMCPy contributors who then do not need to compute Jacobians for their transforms. May also be computationally cheaper i.e. Gaussian Jacobian requires PDF and inverse CDF computations, where the second equation would save the inverse CDF computation.}

\fi 


\end{document}

